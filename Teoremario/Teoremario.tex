\documentclass[twoside,letterpaper,openright,twocolumn, final,11pt]{book}
%landscape Para poner el documento en hojas horizontales
%\usepackage[top = 1.65cm, bottom =1cm,left = 1.9cm,right = 1cm]{geometry} 

\usepackage[scale=0.9]{geometry} 


\usepackage{amsmath,amssymb,amsthm}
\usepackage{latexsym}\usepackage{ amssymb }
\usepackage[spanish]{babel}
\usepackage[charter]{} 
\usepackage[ascii]{inputenc}
\usepackage[T1]{fontenc}
%\usepackage[bitstream-charter]{mathdesign} 
\usepackage{charter}
\usepackage[expert]{mathdesign}

\usepackage{fancyhdr}%paquete para personalizar encabezados y pie de pagina
\usepackage{cite} % para contraer referencias
\usepackage{emptypage} % Quita encabezados y números de página de las paginas en blanco.
\usepackage{ulem} 
\usepackage[all]{xy}

%%%%%%%%%%%%% COLUMNA %%%%%%%%%%%%%%%%%%%%%%%%%%%%%%%%%%%


\setlength{\columnsep}{.37cm}%The is a way to define the distance between the two columns
%\def\columnseprulecolor{\color{lightgray}}
\setlength{\columnseprule}{1pt}%If you need a line to separate the columns, the following command will do the job:




%%%%%%%%%%%%%%%%%%%%%%%%%%%%%%%%%%%%%%%%%%%%%
%%                                ENCABEZADOS
%%%%%%%%%%%%%%%%%%%%%%%%%%%%%%%%%%%%%%%%%%%%%%%
%
% En lo siguiente, fancyhead sirve para configurar la cabecera, fancyfoot para el pie.
% Justificación: C=centered, R=right, L=left, (nada)=LRC
% Página: O=odd, E=even, (nada)=OE
%\leftmark = información de nivel superior (p.e., capítulo en clase book)
%\rightmark = información de nivel inferior (p.e., sección en clase book)
\usepackage{emptypage} 


\pagestyle{fancy}
%Colores de cabeceras, secciones, etc.
%\newcommand{\colorin}{black}

\renewcommand{\chaptermark}[1]{\markboth{\chaptername\ \thechapter.\mbox{ } #1}{}} %con esta linea aparece el nombre del capitulo en minusculas
\renewcommand{\sectionmark}[1]{\markright{ \thesection\mbox{ } #1}{}} %con esta linea aparece el nombre de las secciones en minusculas. Nota, le puedes poner un punto despues de \thesection para que en el encabezado aparezca con un punto despues del numero de la seccion.

\fancyhf{} % Reseteo el contenido de las cabeceras

% Modifica el ancho de las líneas de cabecera y pie y color a la linea del encabezado
\renewcommand{\headrulewidth}{0pt}
\renewcommand{\footrulewidth}{0pt}
% Modifica el ancho de la cabecera 
%\setlength{\headwidth}{20cm}% Cambia el ancho del encabezado
%\setlength{\headheight}{3pt} %ampliar el valor de altura de la cabecera

%Se definen el formato para las cabeceras
\newcommand{\restauraCabecera}{
%\fancyhead[RO,LE]{\thepage} % Números de página en las esquinas de los encabezados
%\fancyhead[RE]{\chaptername \mbox{} \thechapter} % Números de página en las esquinas de los encabezados
\fancyhead[RO]{{\slshape Secci\'on\rightmark}\;\;\;\;\; {\bf \thepage}} % En las páginas impares, parte derecha del encabezado, aparecerá el nombre de la seccion.
%$\mathsection$ simbolo para capitulos
\fancyhead[LE]{{\bf \thepage}\;\;\;\;\;{\slshape\leftmark}} % En las páginas pares, parte izquierda del encabezado, aparecerá el nombre del capitulo.

%Tambien puedes modificar la fuente con que aparece el capitulo ponendo por ejemplo \bfseries
}


%%%%%%%%%%%%%%%% Cabeceras para capitulos especiales %%%%%%%%%%%%%%%%%%%

\newcommand{\cabeceraEspecialsuper}[1]{
\fancyhead[LE]{{{\bf \thepage}\;\;\;\; \slshape{#1}}}
\fancyhead[RO]{\slshape{#1}\;\;\;\;\; {\bf \thepage}}
%\fancyhead[LE]{{#1}}%\fancyhead[RE]{\textsc{#1}}
}

\newcommand{\cabeceraEspecial}[1]{
\fancyhead[L]{\slshape{#1}}
\fancyhead[R]{\slshape{#1}}
\fancyfoot[L]{\bf \thepage}
\fancyfoot[R]{\bf \thepage}
%\fancyhead[LE]{{#1}}%\fancyhead[RE]{\textsc{#1}}
}


\newcommand{\Izq}[1]{
\fancyhead[L]{\slshape{#1}}
%\fancyhead[R]{\slshape{#1}}
\fancyfoot[L]{\bf \thepage}
\fancyfoot[R]{\bf \thepage}
}

\newcommand{\Der}[1]{
%\fancyhead[L]{\slshape{#1}}
\fancyhead[R]{\slshape{#1}}
\fancyfoot[L]{\bf \thepage}
\fancyfoot[R]{\bf \thepage}
}



\setlength{\headsep}{5pt} %Permite definir la distancia entre la base del encabezado y la parte superior del cuerpo del texto

%\setlength{\topmargin}{25pt} %define la distancia vertical entre el margen superior de impresion y la parte superior del encabezado de la hoja.
%%\headheight : define la altura del encabezado



%%%%%%%%%%%%%%%%%%%%%%%%%%%%%%%%%%%%%%%%%%%%%
%                                           PAQUETES PARA EL TEXTO
%%%%%%%%%%%%%%%%%%%%%%%%%%%%%%%%%%%%%%%%%%%%%

%\setlength{\parindent}{.2cm}%Controla identacion

\usepackage{bold-extra} %Combinar \bf con \sc no tiene ningún efecto sin cargar nuevos paquetes. Con \usepackage{bold-extra} expresiones como \textbf{\textsc{versalitas en negritas}} dan el resultado deseado.
%Para obtener negrita itálica se puede usar \textbf{\itshape Negrita itálica}

%\usepackage{helvet}
%\usepackage{mathpazo} 
%\usepackage{bookman}
%\usepackage[scaled]{uarial}
%\usepackage{helvet}
%\usepackage{charter}
%\usepackage{courier}
%\usepackage{chancery}
%\usepackage{lmodern}
%\newfont{\tap}{tap scaled 1200}
%\usepackage[light,condensed,math]{anttor}
%\usepackage[T1]{fontenc}
%\usepackage{helvet} %texto delgado

%\usepackage{concrete}
%\usepackage{pxfonts} %simbolos muy extraños
%\usepackage{txfonts} %texto muy pequeño
%\usepackage{concrete}
%\usepackage{cmbright} % chafita
%\usepackage{fourier}
%\usepackage{pslatex,concrete}

%\usepackage{mathpazo}
%\usepackage{bookman}
%\usepackage{mathptmx}
%\usepackage{newcent}
%\usepackage{calligra}

%\usepackage{pbsi}
%\usepackage{palatino}
%\usepackage{times}
%\usepackage{pslatex}

%%%%%%%%%%%%%%%%%%%%%%%%%%%%%%%%%%%%%%%%%%%%%




%%%%%%%%%%%%%%%%%%%%%%%%%%%%%%%%%%%%%%%%%%%%%
%                        NOTAS AL  PIES DE PAGINA
%%%%%%%%%%%%%%%%%%%%%%%%%%%%%%%%%%%%%%%%%%%%%

%
% Por defecto, las referencias se hacen con números. Si queremos otro tipo de numeración, podríamos haber puesto (antes de \begin{document}) alguno de los siguientes comandos:

%\renewcommand{\thefootnote}{\fnsymbol{footnote}} % numeración por símbolos
%\renewcommand{\thefootnote}{\roman{footnote}} % i, ii, iii...
%\renewcommand{\thefootnote}{\Roman{footnote}} % I, II, III...
%\renewcommand{\thefootnote}{\alph{footnote}} % a, b, c...
%\renewcommand{\thefootnote}{\Alph{footnote}} % A, B, C...
\renewcommand{\thefootnote}{\arabic{footnote}} % 1, 2, 3... (la que hay por defecto)



%Otra cosa importante. La numeración de las notas al pie de página, se resetea con cada capítulo, empezando a numerar otra vez con 1. Si queremos que esto no suceda y que la numeración continue con la del capítulo anterior, podemos poner las siguientes líneas (antes de \begin{document}):
\usepackage{chngcntr}
\counterwithout{footnote}{chapter}

%\renewcommand{\thefootnote}{\fnsymbol{footnote}}
\renewcommand{\thefootnote}{\roman{footnote}} %	Roman numerals (lowercase), e.g., i, ii, iii...
%\renewcommand{\thefootnote}{\Roman{footnote}} %	Roman numerals (uppercase), e.g., I, II, III...

%%%%%%%%%%%%%% %%%%%%%%%%%%%%%%%%%%%%%%%%%%%%%


%%%%%%%%%%%%%%%%%%%%%%%%%%%%%%%%%%%%%%%%%%%%%%%%

%%%%%%%%%%%%%% %%%%%%%%%%%%%%%%%%%%%%%%%%%%%%%
%                                SUAVISADO Y ESPACIADOS
%%%%%%%%%%%%%% %%%%%%%%%%%%%%%%%%%%%%%%%%%%%%%

\sloppy % suaviza las reglas de ruptura de líneas de LaTeX


\frenchspacing % usar espaciado normal después de '.'
%\pagestyle{headings} % páginas con encabezado y pie básico


%-------------------------------------------------------------------------------------------------------------------------------      %                             ESPACIADO DE PARRAFO
%-------------------------------------------------------------------------------------------------------------------------------
%\parskip=4pt% Con carácter general la distancia entre párrafos está controlada por la variable \parskip que es una medida y por lo tanto se modifica simplemente asignándole un nuevo valor.

%-------------------------------------------------------------------------------------------------------------------------------      

%-------------------------------------------------------------------------------------------------------------------------------      %                             ESPACIADO DE TEXTO
%-------------------------------------------------------------------------------------------------------------------------------
%Aumentar el espacio entre líneas (útil para correcciones)
%Para f = 1 => espaciado normal. 
%Para f = 1.3 => 1.5 espacios. 
%Para f = 1.6 => doble espacio.
\linespread{1}
%---------------------------------------------------------------------------------------------------------------------------------

%\setlength{\proofpreskipamount}{10mm}%define un espacio de $x$ milímetros entre el texto anterior al ambiente

%\setlength{\theorempostskipamount}{10mm}%define un espacio de $x$ milímetros entre el texto siguiente al ambiente
%%%%%%%%%%%%%%%%%%%%%%%%%%%%%%%%%%%%%%%%%%%%%%%%



%%%%%%%%%%%%%%%%%%%%%%%%%%%%%%%%%%%%%%%%%%%%%%%%
%                       CONFIGURACION ENUMERATE
%%%%%%%%%%%%%%%%%%%%%%%%%%%%%%%%%%%%%%%%%%%%%%%%

%\renewcommand{\labelenumi}{(\arabic{enumi})}  % (1., 2., 3.,...)
%\renewcommand{\labelenumi}{\alph{enumi})} %(i., ii., iii.,...)
%\renewcommand{\theenumi}{\Roman{enumi}} % (I., II., III.,...)
%\renewcommand{\labelenumi}{\alph{enumi}.}    %(a., b., c.,...)
\renewcommand{\labelenumi}{({\bf \alph{enumi}})}   %[(a), (b), (c),...]
%\renewcommand{\labelenumi}{\Alph{enumi}.}    %(A., B., C.,...)

%%%%%%%%%%%%%%%%%%%%%%%%%%%%%%%%%%%%%%%%%%%%%%%%

%%%%%%%%%%%%%%%%%%%%%%%%%%%%%%%%%%%%%%%%%%%%%%%%
%                          			             CAPITULOS
%%%%%%%%%%%%%%%%%%%%%%%%%%%%%%%%%%%%%%%%%%%%%%%%
\makeatletter
\def\thickhrulefill{\leavevmode \leaders \hrule height 0ex \hfill \kern \z@} %anchura de la linea que rodea a CAPITULO
\def\@makechapterhead#1{
 \vspace*{-30\p@} %Espacio entre el titulo del capitulo y la cabecera
 {\parindent \z@ \centering \reset@font
   %\quad  %linea que rodea a capituloal
   \large \scshape \@chapapp{} \thechapter %aqui modificas: CAPITULO #
    \thickhrulefill  %linea que rodea a capitulo
    \par\nobreak
    %\interlinepenalty\@M
    %\hrule                 %linea que rodea a nombre de capitulo
    \vspace*{1\p@}
     \Large \bf \scshape #1 %aqui modificas: el nombre del capitulo
    \thickhrulefill  %linea que rodea a capitulo
    \vspace*{5\p@}
    %\hrule                  %linea que rodea a nombre de capitulo
  \vskip 5\p@
 }}
%el codigo siguente modifica el formato de los capitulos que forman el precontenido como el indice, el glosario, ...
\def\@makeschapterhead#1{%
 \vspace*{-30\p@} 
 {\parindent \z@ \centering \reset@font
  \thickhrulefill
    \par\nobreak
    \vspace*{1\p@}
   \vspace*{1\p@}
   \Large \bf  \scshape #1% tenia esto: \Huge \bfseries #1\par\nobreak
    \thickhrulefill 
    \vspace*{5\p@}
   \vskip 8\p@

 }}




%%%%%%%%%%%%%%%%%%%%%%%%%%%%%%%%%%%%%%%%%%%%%%								SECCIONES
%%%%%%%%%%%%%%%%%%%%%%%%%%%%%%%%%%%%%%%%%%%%%
\makeatletter
\def\section{\@ifstar\unnumberedsection\numberedsection}
\def\numberedsection{\@ifnextchar[%]
  \numberedsectionwithtwoarguments\numberedsectionwithoneargument}
\def\unnumberedsection{\@ifnextchar[%]
  \unnumberedsectionwithtwoarguments\unnumberedsectionwithoneargument}
\def\numberedsectionwithoneargument#1{\numberedsectionwithtwoarguments[#1]{#1}}
\def\unnumberedsectionwithoneargument#1{\unnumberedsectionwithtwoarguments[#1]{#1}}
\def\numberedsectionwithtwoarguments[#1]#2{%
  \ifhmode\par\fi
  \removelastskip
  \vskip 5ex\goodbreak
  \refstepcounter{section}%
  \hbox to \hsize{\vbox{%
      \noindent
      \leavevmode
      \begingroup
      \Large\bfseries\raggedleft
      \thesection.\ 
      #2\par
      \endgroup
      \vskip -2ex
      \noindent\hrulefill
      \vskip -2.2ex\nobreak
      \noindent\hrulefill
      }}\nobreak
  \vskip 2ex\nobreak
  \addcontentsline{toc}{section}{%
    \protect\numberline{\thesection}%
    #1}%
  }
\def\unnumberedsectionwithtwoarguments[#1]#2{%
  \ifhmode\par\fi
  \removelastskip
  \vskip 5ex\goodbreak
%  \refstepcounter{section}%
  \hbox to \hsize{\vbox{%
      \noindent
      \leavevmode
      \begingroup
     \centering \large \bf\scshape%\raggedleft
%      \thesection.\ 
      #2\par
      \endgroup
      \vskip -2.15ex
      \noindent\hrulefill
      \vskip -2.32ex\nobreak
      \noindent\hrulefill
      }}\nobreak
  \vskip -1ex\nobreak
  \addcontentsline{toc}{section}{%
%    \protect\numberline{\thesection}%
    #1}%
  }
\makeatother

%%%%%%%%%%%%%%%SUBSECTION%%%%%%%


\makeatletter
\renewcommand\subsection{%
 \@startsection{section}{3}{\z@}{3.25ex  \@plus2ex \@minus.18ex }%
  {-.5em}{\centering \bf \scshape \ul}}
\makeatother
%\usepackage{picins}
%\makeatletter
%\def\section{\@ifstar\unnumberedsection\numberedsection}
%\def\numberedsection{\@ifnextchar[%]
%  \numberedsectionwithtwoarguments\numberedsectionwithoneargument}
%\def\unnumberedsection{\@ifnextchar[%]
%  \unnumberedsectionwithtwoarguments\unnumberedsectionwithoneargument}
%\def\numberedsectionwithoneargument#1{\numberedsectionwithtwoarguments[#1]{#1}}
%\def\unnumberedsectionwithoneargument#1{\unnumberedsectionwithtwoarguments[#1]{#1}}
%
%\def\numberedsectionwithtwoarguments[#1]#2{%
%  \ifhmode\par\fi
%  \removelastskip
%  \vskip 3ex\goodbreak
%  \refstepcounter{section}%
%  \hbox to \hsize{%
%    \fbox{%
%      \hbox to 1cm{\hss\bfseries\large\thesection.\ }%
%      \vtop{%
%        \advance \hsize by -1cm
%        \advance \hsize by -2\fboxrule
%        \advance \hsize by -2\fboxsep
%        \parindent=0pt
%        \leavevmode\raggedright\bfseries\large
%        #2
%        }%
%      }}\nobreak
%  \vskip 1mm\nobreak
%  \addcontentsline{toc}{section}{%
%    \protect\numberline{\thesection}%
%    #1}%
%  \ignorespaces
%  }
%  
%\def\unnumberedsectionwithtwoarguments[#1]#2{%
%  \ifhmode\par\fi
%  \removelastskip
%  \vskip 1ex\goodbreak
%%  \refstepcounter{section}%
%   \hbox to \hsize{%
%    \fbox{%
%%      \hbox to 1cm{\hss\bfseries\Large\thesection.\ }%
%      \vtop{%
%      % \advance \hsize by -1cm
%       \advance \hsize by -2\fboxrule
%       \advance \hsize by -2\fboxsep
%        \parindent=0pt
%  \leavevmode\raggedright\bfseries\large
%        #2
%        }%
%      }}\nobreak
%  \vskip 1mm\nobreak
%  \addcontentsline{toc}{section}{%
%%    \protect\numberline{\thesection}%
%    #1}%
%  \ignorespaces
%  }
%\makeatother
%
%%%%%%%%%%%%%%%%%%%%%%%%%%%%%%%%%%%%%%%%%%%%






%%%%%%%%%%%%%%%%%%%%%%%%%%%%%%%%%%%%%%%%%%%%%%%%
%                                 TEOREMAS
%%%%%%%%%%%%%%%%%%%%%%%%%%%%%%%%%%%%%%%%%%%%%%%%
%\theoremstyle{theorem}
\theoremstyle{plain}

\newtheorem{teo}{Teorema}
\newtheorem{dem}[teo]{Demostrac\'ion}
\newtheorem{teorema}[teo]{Teorema}
\newtheorem{propiedades}[teo]{Propiedades}
\newtheorem{prop}[teo]{Proposici\'on}
\newtheorem{afi}[teo]{Afirmaci\'on}
\newtheorem{coro}[teo]{Corolario}
\newtheorem{corolario}[teo]{Corolario}
\newtheorem{afirmacion}[teo]{Afirmaci\'on}
\newtheorem{lema}[teo]{Lema}
\newtheorem{resultado}[teo]{Resultado}%
\newtheorem{proposicion}[teo]{Proposici\'on}







\theoremstyle{definition}

\newtheorem{defi}[teo]{Def\mbox{}inici\'on}

%%%%%%%%%%%%%%%%%%%%%%%%%%%%%%%%%%%%%%%%%%%%%%%%
%                               REDEFINICIONES
%%%%%%%%%%%%%%%%%%%%%%%%%%%%%%%%%%%%%%%%%%%%%%%%
%El sımbolo al final de la demostraci ́n se puede cambiar con el comando:

%\renewcommand{\qedsymbol}{$\boxtimes$}
%\renewcommand{\qedsymbol}{\checkmark}
%\renewcommand{\qedsymbol}{$\lhd$}
%\renewcommand{\qedsymbol}{$\diamonddiamond$}
%\renewcommand{\qedsymbol}{$\boxdot$}
%\renewcommand{\qedsymbol}{\hbox{\boldmath $ \square$ \unboldmath}}
%\renewcommand{\qedsymbol}{$\dagger$}
%\renewcommand{\qedsymbol}{\kreuz}
%\renewcommand{\qedsymbol}{\maltese}
%\renewcommand{\qedsymbol}{$\skull$}
%\renewcommand{\qedsymbol}{\Smiley}
%\renewcommand{\qedsymbol}{\hbox{\boldmath $ \wasylozenge$ \unboldmath}}
%\renewcommand{\qedsymbol}{\small $\blacksquare$}
%\renewcommand{\qedsymbol}{$\diamonddiamond$}
%\renewcommand{\qedsymbol}{\small  \Cross}
%\renewcommand{\qedsymbol}{\small  \CrossOpenShadow}CrossOutline
%\renewcommand{\qedsymbol}{\small  \CrossOutline}
%\renewcommand{\qedsymbol}{\small  \FourStarOpen}

%\renewcommand{\qedsymbol}{\small  \sc q.e.d.}
%\newcommand{\nombredelnuevocomando}[numerodeargumentos][valorespordefecto]{definición}
%\providenewcommand{\nombredelnuevocomando}[numerodeargumentos][valorespordefecto]{definición}
%\renewcommand{\red}{\black}

%Abreviación para \displaystyle   \'





\usepackage{relsize}
\usepackage{ upgreek }
\usepackage{ mathrsfs }


%%%%%%%%%%%%%%%%%%%%%%%%%%%%%%%%%%%%%%%%%%%%%%%%%



%\newcommand{\al}{\alpha} 
%\newcommand{\}{\beta} 				
%\newcommand{\G}{\gamma}
%%%%%%%%%%%%%%%%%%%%%%%%%%%%%%%%%%%%%%%%%%%%%%%%%



\begin{document}
{\tiny
\begin{teo}
Este es un teorema de prueba
\end{teo}
}
\end{document}